\documentclass[t]{beamer}

\usetheme{Pittsburgh}

%%% Работа с русским языком
\usepackage{cmap}					% поиск в PDF
\usepackage{mathtext} 				% русские буквы в формулах
\usepackage[T2A]{fontenc}			% кодировка
\usepackage[utf8]{inputenc}			% кодировка исходного текста
\usepackage[english,russian]{babel}	% локализация и переносы

%% Beamer по-русски
\newtheorem{rtheorem}{Теорема}
\newtheorem{rproof}{Доказательство}
\newtheorem{rexample}{Пример}

%%% Дополнительная работа с математикой
\usepackage{amsmath,amsfonts,amssymb,amsthm,mathtools} % AMS
\usepackage{icomma} % "Умная" запятая: $0,2$ --- число, $0, 2$ --- перечисление

%% Перенос знаков в формулах
\newcommand*{\hm}[1]{#1\nobreak\discretionary{}
{\hbox{$\mathsurround=0pt #1$}}{}}

%%% Работа с картинками
\usepackage{graphicx}  % Для вставки рисунков
\graphicspath{{images/}{tables/}}  % папки с картинками
\setlength\fboxsep{3pt} % Отступ рамки \fbox{} от рисунка
\setlength\fboxrule{1pt} % Толщина линий рамки \fbox{}
\usepackage{wrapfig} % Обтекание рисунков текстом

%%% Работа с таблицами
\usepackage{array,tabularx,tabulary,booktabs} % Дополнительная работа с таблицами
\usepackage{longtable}  % Длинные таблицы
\usepackage{multirow} % Слияние строк в таблице

%%% Программирование
\usepackage{etoolbox} % логические операторы

%%% Другие пакеты
\usepackage{lastpage} % Узнать, сколько всего страниц в документе.
\usepackage{soul} % Модификаторы начертания
\usepackage{csquotes} % Еще инструменты для ссылок
%\usepackage[style=authoryear,maxcitenames=2,backend=biber,sorting=nty]{biblatex}
\usepackage{multicol} % Несколько колонок

%%% Картинки
\usepackage{tikz} % Работа с графикой
\usepackage{pgfplots}
\usepackage{pgfplotstable}

\usepackage{comment}

%%-----------------------------------------------------------------------------------
%%---------------------------------------TITLE---------------------------------------
%%-----------------------------------------------------------------------------------

\title{Байесовская оптимизация на основе апроксимации с помощью нейронной сети	}
\author{Лисов Роман}
\date{\today}
\institute[МФТИ]{Московский физико-технический институт\\ (государственный университет)}


%%-----------------------------------------------------------------------------------
%%---------------------------------------BEGIN---------------------------------------
%%-----------------------------------------------------------------------------------

\begin{document}

\begin{comment} 
 %---------------------COMMENT!!!!!------------------------------
\end{comment}


\frame[plain]{\titlepage}	% Титульный слайд

%%-----------------------------------------------------------------------------------
%%---------------------------------------SECTION-------------------------------------
%%-----------------------------------------------------------------------------------

\section{Постановка задачи}

\begin{frame}
	\frametitle{\insertsection}
	\framesubtitle{\insertsubsection}
	
	\begin{itemize}
		\item \alert{Автор задачи} --- Максим Панов
		\item \alert{Цель работы} --- Построить модель байесовского оптимизатора на основе работы библотеки GPyOpt, аппроксимируя при этом функцию с помощью нейронной сети
	\end{itemize}
\end{frame}
 
%%-----------------------------------------------------------------------------------
%%---------------------------------------SECTION-------------------------------------
%%-----------------------------------------------------------------------------------

\section{Этапы решения} 
\begin{frame}
	\frametitle{\insertsection}
	\framesubtitle{\insertsubsection}
	
	\begin{enumerate}
		\item Изучение математического аппарта
		\begin{itemize}
			\item Гауссовских процессов
			\item Байсовской (Surrogate) оптимизации
			\item Метод Dropout 
		\end{itemize}
		\item Построение модели
		\begin{itemize}
			\item Моделирование функции для оптимизации
			\item Аппроксимация функции с помощью NN
			\item Применение метода Dropout к NN
			\item Построение модели оптимизатора с помощью библиотеки GPyOpt
			\item Применение данных полученных из сетки в оптимизаторе
		\end{itemize}
		\item Результаты
		\begin{itemize}
			\item Сравнение результатов оптимизации с применением данных выхода сетки и без него
			\item Вывод о качестве работы метода DropOut
		\end{itemize}
	\end{enumerate}
\end{frame}

%%-----------------------------------------------------------------------------------
%%---------------------------------------SECTION-------------------------------------
%%-----------------------------------------------------------------------------------

\section{Теоретическое введение}

\subsection{Гауссовские процессы}

\begin{frame}
\frametitle{\insertsection}
\framesubtitle{\insertsubsection}
\begin{columns}[T] % contents are top vertically aligned
\begin{column}[T]{6cm} % each column can also be its own environment
\center{Prior distribution on picture (a):}
\includegraphics[width = \linewidth]{prior}
\end{column}
\begin{column}[T]{6cm}
\center{Posterior distribution on picture (b):}
\includegraphics[width = \linewidth]{posterior}
\end{column}
\end{columns}
\end{frame}


\begin{frame}
\frametitle{\insertsection}
\framesubtitle{\insertsubsection}
The posterior distribution over the weights computed by Bayes's rule:
$$ posterior = \frac{likelihood \cdot{ prior }}{marginal likelihood} $$\pause

\%Then: \\
---Gaussian posterior\\
---Definiiton of GP\\
---Prediction with noise free and noisy observations\\
\end{frame}




%-----------------------------------------------------------------------------------%

\subsection{Surrogate optimization}

\begin{frame}
\frametitle{\insertsection}
\framesubtitle{\insertsubsection}
\begin{columns}[T] % contents are top vertically aligned
\begin{column}[T]{6cm} % each column can also be its own environment
\center{Example of a function to optimize:}
\includegraphics[width = \linewidth]{function}
\end{column}
\begin{column}[T]{6cm}
\center{Mean squared error:}
\includegraphics[width = \linewidth]{S_(x)}
\end{column}
\end{columns}
\end{frame}

\begin{frame}
\frametitle{\insertsection}
\framesubtitle{\insertsubsection}
\begin{columns}[T] % contents are top vertically aligned
\begin{column}[T]{6cm} % each column can also be its own environment
\center{Probability of improvement:}
\includegraphics[width = \linewidth]{function}
\end{column}
\begin{column}[T]{6cm}
\center{Expected improvement:}
\includegraphics[width = \linewidth]{S_(x)}
\end{column}
\end{columns}

\%Then:\\
---Formulas

\end{frame}

%-----------------------------------------------------------------------------------%


\subsection{Обучение сетки, апроксимация фунцкии с помощью метода DropOut}

\begin{frame}
\frametitle{\insertsection}
\framesubtitle{\insertsubsection}

\begin{figure}[H]
	\centering
	\includegraphics[width=0.9\linewidth]{dropout} 
\end{figure}
\end{frame}


%%-----------------------------------------------------------------------------------
%%---------------------------------------SECTION-------------------------------------
%%-----------------------------------------------------------------------------------
\section{Построение модели}


\subsection{Рассмотрение интересных функций для оптимизации}
\begin{frame}
\frametitle{\insertsection}
\framesubtitle{\insertsubsection}


Примеры:
\begin{itemize}
	\item двумерная функция $Rosenbrock$
	\item многомерная функция $RosenbrockND$
	\item функция $Forrester$
\end{itemize}

Идея: Оптимизиация на основе \alert{гауссовских процессов} существенно усложняется с ростом размерности и объёма выборки. Попробуем улучшить данную ситуацию с помощью апроксимации, проведённой нейронной сетью заранее.

\end{frame}

\begin{frame}
\frametitle{\insertsection}
\framesubtitle{\insertsubsection}
\begin{figure}[H]
\center{Функция розенброка для двумерноего случая:}
	\centering
	\includegraphics[width=0.6\linewidth]{rosen_2D} 
\end{figure}


\end{frame}

%-----------------------------------------------------------------------------------%

\subsection{GpyOpt}

\begin{frame}
\frametitle{\insertsection}
\framesubtitle{\insertsubsection}

Something about this library...

\end{frame}



%%-----------------------------------------------------------------------------------
%%---------------------------------------SECTION-------------------------------------
%%-----------------------------------------------------------------------------------
\section{Результаты}

\subsection{Сравнение моделей}

\begin{frame}
\frametitle{\insertsection}
\framesubtitle{\insertsubsection}
\center{Результаты оптимизации с использование выхода сетки:}
\begin{center}
\includegraphics[width = 0.5\linewidth]{bo_with_NN}
\end{center}
\center{Результаты оптимизации только на основе ГП:}
\begin{center}
\includegraphics[width = 0.5\linewidth]{bo_without_NN}
\end{center}

\end{frame}

%%-----------------------------------------------------------------------------------

\subsection{Плюсы и минусы модели}
\begin{frame}
\frametitle{\insertsection}
\framesubtitle{\insertsubsection}

Плюсы:
\begin{itemize}
	\item ---
	\item ---
	\item ---
\end{itemize}

Минусы:
\begin{itemize}
	\item ---
	\item ---
\end{itemize}

\end{frame}


%%-----------------------------------------------------------------------------------
%%---------------------------------------SECTION-------------------------------------
%%-----------------------------------------------------------------------------------

\section{Спасибо за внимание!}
\begin{frame}
\frametitle{\insertsection}
\end{frame}			


\end{document}